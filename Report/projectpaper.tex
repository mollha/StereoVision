\documentclass[12pt,a4paper]{article}
\usepackage{times}
\usepackage{durhampaper}
\usepackage{url}
\usepackage{harvard}
\usepackage{cite}
\usepackage{graphicx}


\citationmode{abbr}
\bibliographystyle{plain}
\graphicspath{ {./images/} }

\title{Using Stereo Vision for Object Distance Ranging}
\author{} % leave; your name goes into \student{}
\student{Molly Hayward}

\date{20/11/2019}

\begin{document}

\maketitle

\vspace{3 mm}
\section{Introduction}
\noindent Producing accurate depth-estimation of objects is a complex problem in computer vision, as images often contain noise due to inconsistent illumination, object occlusion and challenging weather conditions. In this report, I detail my approach to integrating state-of-the-art object detection (YOLO) with dense stereo ranging, and a high-level overview of this solution is given in Figure 1.  I experimented with different implementations of stages 3-7 in order to improve performance under challenging conditions, and provide a comparative evaluation of these techniques in the remainder of the report.


\begin{center}
	\includegraphics[scale=0.25]{Solutiondiagram}
	\textit{Figure 1: Solution overview}
\end{center}


\newpage



\section{Solution Design}
\paragraph{Stereo image pre-processing}
In pre-process the stereo image pair in order to improve subsequent disparity calculations
\includegraphics[scale=2]{no_filter}


\paragraph{Disparity post-processing}
The disparity map is initially texturized and full of holes


\section{Evaluation}
The run-time of my solution is ...
Most objects are detected ....

\bibliography{projectpaper}


\end{document}